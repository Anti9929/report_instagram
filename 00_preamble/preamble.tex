% begin of preamble

% define general document properties
\documentclass[12pt,a4paper]{article}	% peut remplacer article par "report"," book"
% si report, plus joli de mettre le package "fnychap" et utiliser "chapter" comme premier (et non pas "section")
	%\usepackage[Lenny]{fncychap}	% Lenny, Sonny, PetersLenny, Glenn, Conny, Rejne, Bjarne, Bjornstrup
\usepackage[utf8]{inputenc} % make content UTF8
%\usepackage[latin1]{inputenc} % make content ISO-8859-1
%\usepackage[T1]{fontenc} % to use with latin1, gives all the umlaute

%define borders of page like in word :)
\usepackage{geometry} 
\geometry{a4paper,left=25mm, right=25mm, top=30mm, bottom=30mm}

% use for math stuff
\usepackage{amsmath}
\usepackage{amsfonts}
\usepackage{amssymb}
\usepackage{algorithm}
\usepackage{algorithmic}
\usepackage{longtable}
\usepackage{tabu}
\renewcommand{\algorithmicrequire}{\textbf{Input:}}
\renewcommand{\algorithmicensure}{\textbf{Output:}}
%add the "absolute value" and the "norm value" signs
%\providecommand{\abs}[1]{\lvert#1\rvert}
%\providecommand{\norm}[1]{\lVert#1\rVert}
\usepackage[toc,page]{appendix}
% select the language for the report
%\usepackage[ngerman, english, francais]{babel} % LAST language (english) is the main language (table of contents, annexes, ...); other languages can be used (special characters...)

% macro definition for exponential and unit formatting
\newcommand{\e}[1]{\ensuremath{\times 10^{#1}}}
\newcommand{\unit}[1]{\ensuremath{\, \mathrm{#1}}}

% configure graphics
\usepackage{graphicx}
\usepackage{caption}
\usepackage{subcaption}
\graphicspath{{Images/}} % Path Where images reside
\usepackage[process=auto,crop=pdfcrop]{pstool}	% by Basil for the matlabfrag script
\usepackage{float}
%\DeclareGraphicsExtensions{.png,.jpeg,.pdf,.jpg} % Valide image format extensions

% allows the coloration of text and defines custom colors
\usepackage{color}
 
\definecolor{darkblue}{rgb}{0, 0, 0.6}
\definecolor{darkgreen}{rgb}{0, 0.6, 0}
\definecolor{darkred}{rgb}{0.7, 0, 0}
\definecolor{gold}{rgb}{0.6, 0.6, 0}

% % allows to use slasboxes in tables
% \usepackage{slashbox}

% use external pdf pages
\usepackage{pdfpages}

%  setup multicolumn and float
\usepackage{multicol} % Multiple Columns
\usepackage{multirow}
%\usepackage{subfig} % Use modern subfig instead of subfigure

%configure citations
\usepackage[numbers, comma]{natbib}


% use watermarks to mark draft
%\usepackage[firstpage]{draftwatermark}		% only the first page
%\usepackage{draftwatermark}		% all pages
%\SetWatermarkScale{5}
%\SetWatermarkLightness{0.8}
%\SetWatermarkText{!DRAFT!}

% make subsubsubsection(=parapraph) come real on table of contents and be numerated
\setcounter{secnumdepth}{4}
\setcounter{tocdepth}{4}

%adds the EURO-symbol: \euro
\usepackage{eurosym}

%new definition of table, where you can adjust the width of a cell: \begin{tabular}{|C{2cm}|}
\usepackage{tabularx}
\newcolumntype{C}[1]{>{\centering\arraybackslash}p{#1}}
\newcolumntype{R}[1]{>{\raggedright\arraybackslash}p{#1}}
\newcolumntype{L}[1]{>{\raggedleft\arraybackslash}p{#1}} 

% make hyperlinks of table of contents to sections, references (bibliography), URLs
 \usepackage{hyperref}
 \hypersetup{
 	colorlinks=true,
	linkcolor = darkblue,
	citecolor = darkgreen,
	filecolor = gold,
	urlcolor = darkred,
	linktoc = section,
 }

 \usepackage{floatflt}
% correct bad hyphenation here
\hyphenation{}

%name the pictures and formulas with the corresponding chapter
%\numberwithin{equation}{section}
%\numberwithin{figure}{section}
%\numberwithin{table}{section}
%\numberwithin{footnote}{section}

\usepackage{wrapfig}


%\usepackage{subfigure}

%define the degree(in angle) and degree celsius symbols (�C)
\usepackage{xspace}
\newcommand{\degree}{\ensuremath{\,^{\circ}}\xspace}
\newcommand{\degreeCelsius}{\ensuremath{\,^{\circ}\mathrm{C}}\xspace}

%Color, example for only a section: {\color{red} This is very important!}
\usepackage{color}

%adds a lot of the symbols of http://www.ctan.org/tex-archive/info/symbols/comprehensive/symbols-a4.pdf
\usepackage{pifont}

%adds the "born" and "died" symbols: \textborn, \textdied
\usepackage{textcomp}

%my own page number counter
\newcounter{romanPagenumber}


%%%%  enlever la num�rotation des r�f�rences dans la bibliographie
% \makeatletter
%  \renewcommand\@biblabel[1]{}
%  \makeatother
 

 %for the summary
 
 %setup multicolumn and float
\usepackage{multicol} % Multiple Columns
%\usepackage{subfig} % Use modern subfig instead of subfigure

%configure citations
%\usepackage[round]{natbib}
%\usepackage[twoside,citeonce(page),citeonce(chapter)]{footbib}



%Pour les en-tetes
\usepackage{fancyhdr}
\pagestyle{fancy}
%\renewcommand\headrulewidth{5pt}
\fancyhead[L]{\footnotesize{EPF Lausanne \\HCI
}}
\fancyhead[C]{\footnotesize{}}
\fancyhead[R]{\footnotesize{\# Travel\\}}
	%Pour avoir le numero de page 1/ x
	%\usepackage{lastpage}
	%\fancyfoot[C]{\thepage/\pageref{LastPage}}
%\fancyfoot[C]{Rubrique \LaTeX}
%\fancyfoot[R]{\today}

\usepackage{url}
 
 
\usepackage{listings} 

\definecolor{colKeys}{rgb}{0,0,1} 
\definecolor{colIdentifier}{rgb}{0,0,0} 
\definecolor{colComments}{rgb}{0,0.7,0} 
\definecolor{colString}{rgb}{0.6,0.1,0.1} 

\lstset{%configuration de listings 
float=hbp,% 
basicstyle=\ttfamily\small, % 
identifierstyle=\color{colIdentifier}, % 
keywordstyle=\color{colKeys}, % 
stringstyle=\color{colString}, % 
commentstyle=\color{colComments}, % 
columns=flexible, % 
tabsize=2, % 
frame=simple, % none, shadowbox, simple, double, lines
frameround=tttt, % 
extendedchars=true, % 
showspaces=false, % 
showstringspaces=false, % 
numbers=left, % none, left
numberstyle=\tiny, % 
breaklines=true, % 
breakautoindent=true, % 
captionpos=b,% 
xrightmargin=1cm, % 
xleftmargin=1cm 
} 

%% include graphical storage
\graphicspath{{./images/}{./images_tpx/}}

\EndPreamble
% end of preamble
% ----------------------------------------------------------------------------------------------------------------
